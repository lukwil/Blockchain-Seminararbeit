% !TeX spellcheck = de-DE
% !TeX encoding = utf8
% !TeX program = lualatex
% !TeX TXS-program:compile = txs:///lualatex/[--shell-escape]
% !BIB program = biber
% -*- coding:utf-8 mod:LaTeX -*-

% vv  scroll down to line 200 for content  vv


\let\ifdeutsch\iftrue
\let\ifenglisch\iffalse
\input{pre-documentclass}
\documentclass[
  fontsize=12pt, % Vorgabe von Herrn Müller
  a4paper,  % Standard format - only KOMAScript uses paper=a4 - https://tex.stackexchange.com/a/61044/9075
  oneside,  % oneside für einseitigen Druck, twoside für buchartigen druck
  bibliography=totoc,
                 %idxtotoc,   %Index ins Inhaltsverzeichnis
  %               liststotoc, %List of X ins Inhaltsverzeichnis, mit liststotocnumbered werden die Abbildungsverzeichnisse nummeriert
  headsepline,
  cleardoublepage=empty,
  parskip=half,
  %               draft    % um zu sehen, wo noch nachgebessert werden muss - wichtig, da Bindungskorrektur mit drin
  draft=false
]{scrbook}
\input{config}
%\setcounter{tocdepth}{4}
%%%
% EN: Syntax highligthing using pygments package
\usepackage[chapter]{minted}
\usepackage{titlesec}

%\usepackage[ngerman]{babel}% deutsche Trennregeln
%\usepackage[T1]{fontenc}% wichtig für Trennung von Wörtern mit Umlauten
%\usepackage{microtype}% verbesserter Randausgleich

\usepackage[a4paper,%
  head=15mm,
  headsep=5mm,
  bottom=25mm,
  left=35mm,
  right=15mm, 
  footskip=10mm]{geometry}

\addtokomafont{chapter}{\LARGE} % kleinere Schrift für Kapitelüberschriften

\RedeclareSectionCommand[beforeskip=2sp, afterskip=1sp]{chapter}
\RedeclareSectionCommands[beforeskip=1sp, afterskip=1sp]{section,subsection,subsubsection}
\RedeclareSectionCommands[beforeskip=1sp, afterskip=1sp]{paragraph,subparagraph}

% EN: line numbers within page margins
% DE: Zeilennummern innerhalb vom Rand
\setminted{numbersep=5pt, xleftmargin=12pt, fontsize=\huge, baselinestretch=1}
%%%

%http://www.jevon.org/wiki/Eclipse_Pygments_Style
%\usemintedstyle{eclipse}
%
%\usemintedstyle{autumn}
%\usemintedstyle{rrt}
%\usemintedstyle{borland}
%\usemintedstyle{friendlygrayscale}
\usemintedstyle{friendly}

%EN: compatibility of packages minted and listings with respect to the numbering of "List." caption
%    source: https://tex.stackexchange.com/a/269510/9075
\AtBeginEnvironment{listing}{\setcounter{listing}{\value{lstlisting}}}
\AtEndEnvironment{listing}{\stepcounter{lstlisting}}
%EN: We use the Listing environment to have the nice bar. So, we also have to patch the "Listing" environment for consistent counters
\AtBeginEnvironment{Listing}{\setcounter{listing}{\value{lstlisting}}}
\AtEndEnvironment{Listing}{\stepcounter{lstlisting}}

% Abkürzungsverzeichnis
\input{acronyms}

\makeindex
\usepackage{setspace} % Vorgabe von Herrn Müller
\onehalfspacing
\begin{document}

%----------------------------------------------------------------------------------------
%	TITELSEITE
%----------------------------------------------------------------------------------------
% Im Anhang Credit (CC BY 4.0!) für https://www.overleaf.com/latex/templates/uppsala-university-template/jvjprsfnzgbj#.WzX_c9IzaUk
\begin{titlepage}

\newcommand{\HRule}{\rule{\linewidth}{0.5mm}} % Defines a new command for the horizontal lines, change thickness here

\center % Center everything on the page
 
%----------------------------------------------------------------------------------------
%	HEADING SECTIONS
%----------------------------------------------------------------------------------------

\textsc{\LARGE Studienarbeit}\\[1.5cm] % Name of your university/college
\includegraphics[scale=.15]{logos/Logo_Hochschule_Kempten.png}\\[1cm] % Include a department/university logo - this will require the graphicx package
\textsc{\Large Seminar Blockchain }\\[0.5cm] % Major heading such as course name
%\vspace{2cm}
{\Large Sommersemester 2019} % Date, change the \today to a set date if you want to be precise
%\textsc{\large Course code}\\[0.5cm] % Minor heading such as course title

%----------------------------------------------------------------------------------------
%	TITLE SECTION
%----------------------------------------------------------------------------------------
\vspace{2.5cm}
{ \huge \bfseries Anwendungen von Smart Contracts }% Title of your document
\vspace{2.5cm}
 
%----------------------------------------------------------------------------------------
%	AUTHOR SECTION
%----------------------------------------------------------------------------------------

\begin{minipage}{0.4\textwidth}

\emph{Autor:}\\
Lukas \textsc{Willburger}\\ % Your name
Matrikelnr. 322445 \\
Master Informatik, 1. Semester \\ \newline 


\emph{Seminarleitung: }\\
Prof. Nikolaus \textsc{Steger} \\ \newline  \newline

\end{minipage}\\[1cm]

% If you don't want a supervisor, uncomment the two lines below and remove the section above
%\Large \emph{Author:}\\
%John \textsc{Smith}\\[3cm] % Your name

%----------------------------------------------------------------------------------------
%	DATE SECTION
%----------------------------------------------------------------------------------------



\vfill % Fill the rest of the page with whitespace

\end{titlepage}

\pagenumbering{gobble}
%tex4ht-Konvertierung verschönern
\iftex4ht
  % tell tex4ht to create picures also for formulas starting with '$'
  % WARNING: a tex4ht run now takes forever!
  \Configure{$}{\PicMath}{\EndPicMath}{}
  %$ % <- syntax highlighting fix for emacs
  \Css{body {text-align:justify;}}

  %conversion of .pdf to .png
  \Configure{graphics*}
  {pdf}
  {\Needs{"convert \csname Gin@base\endcsname.pdf
      \csname Gin@base\endcsname.png"}%
    \Picture[pict]{\csname Gin@base\endcsname.png}%
  }
\fi

%\VerbatimFootnotes %verbatim text in Fußnoten erlauben. Geht normalerweise nicht.
\input{commands}


%Eigener Seitenstil fuer die Kurzfassung und das Inhaltsverzeichnis
%\deftripstyle{preamble}{}{}{}{}{}{\pagemark}
%Doku zu deftripstyle: scrguide.pdf
%\pagestyle{preamble}
%\renewcommand*{\chapterpagestyle}{preamble}



%Kurzfassung / abstract
%auch im Stil vom Inhaltsverzeichnis
%\ifdeutsch
%  \section*{Kurzfassung}
%\else
%  \section*{Abstract}
%\fi
%
%... Short summary of the thesis ...
%
%\cleardoublepage


% BEGIN: Verzeichnisse

\iftex4ht
\else
  \microtypesetup{protrusion=false}
\fi

%%%
% Literaturverzeichnis ins TOC mit aufnehmen, aber nur wenn nichts anderes mehr hilft!
% \addcontentsline{toc}{chapter}{Literaturverzeichnis}
%
% oder zB
%\addcontentsline{toc}{section}{Abkürzungsverzeichnis}
%
%%%

%Produce table of contents
%
%In case you have trouble with headings reaching into the page numbers, enable the following three lines.
%Hint by http://golatex.de/inhaltsverzeichnis-schreibt-ueber-rand-t3106.html
%
%\makeatletter
%\renewcommand{\@pnumwidth}{2em}
%\makeatother
%

\tableofcontents

% Bei einem ungünstigen Seitenumbruch im Inhaltsverzeichnis, kann dieser mit
% \addtocontents{toc}{\protect\newpage}
% an der passenden Stelle im Fließtext erzwungen werden.

%Auflistung der Codezeilen
%\listoffigures

%Auflistung der Tabellen
%\listoftables

%Wird nur bei Verwendung von der lstlisting-Umgebung mit dem "caption"-Parameter benoetigt
%\lstlistoflistings
%ansonsten:
%\ifdeutsch
%  \listof{Listing}{Verzeichnis der Listings}
%\else
%  \listof{Listing}{List of Listings}
%\fi

%mittels \newfloat wurde die Algorithmus-Gleitumgebung definiert.
%Mit folgendem Befehl werden alle floats dieses Typs ausgegeben
%\ifdeutsch
%\listof{Algorithmus}{Verzeichnis der Algorithmen}
%\else
  %\listof{Algorithmus}{List of Algorithms}
%\fi
%\listofalgorithms %Ist nur für Algorithmen, die mittels \begin{algorithm} umschlossen werden, nötig
% Abkürzungsverzeichnis

%\printnoidxglossaries
\thispagestyle{empty}
\iftex4ht
\else
  %Optischen Randausgleich und Grauwertkorrektur wieder aktivieren
  \microtypesetup{protrusion=true}
\fi

% END: Verzeichnisse


% Headline and footline
\renewcommand*{\chapterpagestyle}{scrplain}
\pagestyle{scrheadings}
\pagestyle{scrheadings}
\ihead[]{}
\chead[]{}
\ohead[]{\headmark}
\cfoot[]{}
\ofoot[\usekomafont{pagenumber}\thepage]{\usekomafont{pagenumber}\thepage}
\ifoot[]{}



%-----------------------------------------------------------------------------
%
% Main content starts here
%
%-----------------------------------------------------------------------------
\pagenumbering{arabic}
\clearpage
\setcounter{page}{1}
% Hier stehen gesammelt alle Inhaltskapitel drin
\spacing{1.3}
% Hier können die einzelnen Kapitel inkludiert werden. Sie müssen in den 
% entsprechenden .TEX-Dateien vorliegen. Die Dateinamen können natürlich 
% angepasst werden.
\chapter{Einleitung}
\label{chap:Einleitung}
Die Blockchain-Technologie hat sich in den letzten Jahren zu einem regelrechten Hype-Thema entwickelt. So ist es nicht verwunderlich, dass alleine in Berlin im Jahr 2018 bereits 64 Blockchain-Startups niedergelassen waren \cite{AnzahlStartups2018}. Kryptowährungen, wie beispielsweise Bitcoin oder Ethereum, sind meist die erste Assoziation, die der Begriff Blockchain hervorruft. Über deren Sinn und Daseinsberechtigung im Geldmarkt wird seit Beginn angeregt diskutiert. So sagte der amerikanische Großinvestor Warren Buffet 2014 in einem Interview mit dem US-amerikanischen Sender CNBC über Bitcoin: \qq{Stay away from it. It’s a mirage, basically} \cite{Buffett2014}.

Es muss auch angemerkt werden, dass die Blockchain längst nicht mehr ausschließlich für Kryptowährungen bereitsteht. Mit der kontinuierlichen Weiterentwicklung avancierten vor allem die sogenannten \textit{Smart Contracts} zu einem wichtigen Bestandteil der Plattformen \cite[vgl.][S.~8]{Schuette2017}. Diese sind vor allem im Geschäftsleben von großem Interesse.

Werden in einem realen Umfeld Absprachen getroffen, stehen für beide Vertragsparteien Forderungen und zu erbringende Leistungen im Vordergrund. In bestimmten Fällen kann nicht auf deren Einhaltung vertraut werden, wodurch sich der Einsatz von Smart Contracts anbietet. Dadurch wird einerseits durch die technologische Architektur der Blockchain Manipulationssicherheit garantiert, als auch die Nachvollziehbarkeit aller geschäftlichen Transaktionen gewährleistet.

Einige Szenarien sollen in dieser Arbeit beispielhaft anhand von Use Cases der vorgestellten Unternehmen dargelegt werden. 
\chapter{Überblick}
\label{chap:Ueberblick}

\textit{Anmerkung: Da die Grundlagen von Ethereum Smart Contracts Bestandteil eines dieser Arbeit vorausgegangenen Seminarvortrags sind, wird auf deren Definition und weitere Einzelheiten an dieser Stelle nicht weiter eingegangen. Ein Grundlagenwissen zur Thematik gilt als vorausgesetzt.}

Sucht man mit den gängigen Suchmaschinen nach Smart Contract Anwendungen, so fällt auf, dass viele der gelisteten Suchergebnisse lediglich mögliche Szenarien beschreiben, ohne konkrete Anwendungsfälle beziehungsweise anwendende Unternehmen zu nennen. Bei einem genaueren Blick auf die Webseiten bestätigt sich die anfängliche Vermutung, dass viele Personen möglichst viel vom Hype um Blockchain und Bitcoin profitieren wollen. \\
Dennoch lassen sich vielversprechende Szenarien finden, welche auszugsweise in Abbildung \ref{fig:mindmap} mithilfe einer Mindmap dargestellt sind.

\begin{figure}[h!]
  \centering
  \includegraphics[width=\textwidth]{Bilder/Mindmap.png}
  \caption[Mindmap zu Smart Contracts]{Mindmap zu Smart Contracts (eigene Darstellung)}
  \label{fig:mindmap}
\end{figure}

Da es in dem geringen Umfang dieser Seminararbeit nicht möglich ist, im Detail auf alle Bereiche einzugehen, soll nun begründet dargelegt werden, wieso die jeweiligen Beispiele von der näheren Betrachtung ausgeschlossen wurden.\\ 

\textbf{E-Voting}\\
Generell ist das E-Voting ein Bereich, für den Smart Contracts prädestiniert scheinen \cite[vgl. z. B.][]{McCorry2017, Kshetri2018, Yavuz2018}. Die sichere, elektronische Abgabe von Stimmen, ist ein Thema, das viele Länder, darunter die Schweiz, beschäftigt. Diese gilt auf dem Fachgebiet seit Jahren als Pionier. Dort wurde 2018 das E-Voting in der Stadt Zug mittels Blockchain erfolgreich getestet \cite[vgl.][]{luxoft2018}. Mit Ökosystemen wie Agora wird zudem an alltagstauglichen Standards für die elektronische Stimmabgabe gearbeitet \cite[vgl.][]{Agora2019}. Trotzdem wird oftmals noch, wie in Deutschland, auf analoge Art und Weise abgestimmt. Ein Grund dafür stellt die Angreifbarkeit der Systeme dar. Daher wurde von der Schweiz die Offenlegung des Quellcodes und die Einrichtung eines Bug Bounty Programms beschlossen \cite[vgl.][]{Sperlich2019}. Überträgt man diese Gefahren auf die Blockchain, wird die Angriffsfläche durch solch eine basierte Lösung minimiert, generell sind Attacken aber nicht undurchführbar. \\
Aufgrund der allgemeinen Kontroverse über die Anwendung, sowie der Schwierigkeit, genaue Informationen über die eingesetzten Plattformen und Smart Contracts bei Blockchain-basierten Lösungen zu erhalten, wurde dieses Thema nicht zur genaueren Bearbeitung herangezogen.

\textbf{Aktienhandel}\\
Der Aktienhandel ist ein typisches Beispiel für zentralisierte Handelsplattformen. Mittels Smart Contracts und der Blockchain ließen sich an Transaktionen beteiligte Intermediäre entfernen und somit Kosten sparen \cite[vgl.][]{Notheisen2017}.\\
Eine komplett funktionierende Blockchain-Börse gibt es Stand heute nicht. Jedoch nähert sich zum Zeitpunkt des Verfassens dieser Arbeit mit SprinkleXChange die erste dieser Art ihrem Markteintritt \cite[vgl.][]{Hoikkala2019}.\\
Durch den frühen Stand der Implementierungen bzw. der Forschungskonzepte eignet sich dieses Thema nur bedingt für eine tiefe Begutachtung, weshalb es zugunsten anderer Themen nicht betrachtet wird.

\textbf{Urheberrechte}\\
Die Wahrung der Urheberrechte stellt in einer digitalen und globalisierten Welt eine große Herausforderung dar. Mit der umstrittenen EU-Reform um Artikel 13 respektive 17 sollte das Urheberrecht grundlegend erneuert werden \cite[vgl.][]{bpb2019}.\\
Musikverlage, wie Ujo Music, gehen mithilfe der Blockchain andere Wege und vertreiben über eine mehrschichtige Softwarearchitektur die Titel ihrer Künstler dezentral \cite[vgl.][]{Attar2018}. Nach ersten Tests ist die weitere Vorgehensweise jedoch unklar \cite[vgl.][S. 121 f.]{Gilli2019}.\\
Da auch hier die Entwicklungen der Industrie noch nicht absehbar und wenige konkrete Beispiele vorhanden sind, ist es nicht für diese Arbeit geeignet.

\textbf{Juristisches Vertragsmanagement}\\
Das klassische Vertragsmanagement, welches heute von einem Notar geführt wird, kann mittels Smart Contracts vereinfacht werden. Der Markt für Legal Techs, also Startups in der Rechtsbranche, scheint durchaus gegeben zu sein, da die Digitalisierung auch in diesem Bereich eine immer größere Rolle spielt. Eines davon ist die Firma TODO: [EINFÜGEN!!!!!!!!!]\\
Solange aber die rechtlichen Grundlagen für die Rechtssicherheit elektronischer Verträge in der Blockchain nicht gegeben sind, wird ihre Anwendung minimal sein.


\textbf{Banking}\\
Im Zuge der Recherche wurde der Kontakt zu Banken über öffentlich zugängliche E-Mail-Adressen gesucht. Da sich bis zum Zeitpunkt des Verfassens dieser Arbeit kein angeschriebenes Geldinstitut gemeldet hat, sind wenig verfügbare Informationen vorhanden. Allerdings zeigen Vorstöße, wie der von Wirecard \cite[vgl.][]{Weidemann2018}, dass es sich um ein durchaus interessantes Gebiet im Bankensegment handelt. Da Banken aber ohnehin darauf bedacht sein dürften, sensible Informationen wie z.B. Bankdaten nicht öffentlich zu speichern, wird in heutigen und zukünftigen Implementierungen von Smart Contracts wohl keine öffentliche Blockchain, wie beispielsweise Ethereum, herangezogen werden. \\
Das erschwert die Erlangung von Wissen über die eingesetzen Verfahren und Lösungen, weshalb Banking von den Themen ausgeschlossen wird.

\textbf{Gesundheitswesen}\\
TODO!!!

Somit verbleiben die Bereiche Reisen, Versicherung, Hilfsgelder, Energieversorgung/Smart Grid sowie Supply Chain/Logistik zur weiteren Begutachtung.
\chapter{Logistik}
\label{chap:Logistik}

Laut einer Bitkom-Studie aus dem aktuellen Jahr sehen 92 \% der befragten Logistikunternehmen eine Beschleunigung im Transport von Produkten mithilfe der Digitalisierung gegeben \cite{Bitkom2019}. Generell ist in der Branche zunehmen ein Drang nach Digitalisierung aufgrund der komplexen Prozessketten zu sehen. Wo in der Logistik Abmachungen zwischen Vertragsteilnehmern stattfinden, lassen sich Systeme auch ideal mittels Smart Contracts umsetzen. 

\section{Dachser}
Im Vorlauf zu dieser Arbeit und der dazugehörigen Präsentation konnte eine Besprechung mit einem fachkundigen Mitarbeiter der Firma Dachser aus Kempten durchgeführt werden. Seit 2016 wird dort in Zusammenarbeit mit dem Fraunhofer Institut an der Sinnhaftigkeit von Blockchain-Anwendungen sowie Smart Contracts und deren Use Cases geforscht. Dabei werden besonders zwei mögliche Systeme in Betracht gezogen: Hyperledger und Ethereum. Ziel ist es, die Technologie hinter den bekannt gewordenen Schlagworten zu verstehen.

\begin{figure}[h!]
  \centering
  \includegraphics[width=\textwidth]{Bilder/Palettentausch.png}
  \caption[Palettentausch]{Palettentausch \cite{Disponaut2016}}
  \label{fig:palettentausch}
\end{figure}

Ein konkreter Anwendungsfall wurde im Rahmen einer Bachelorarbeit im Wintersemester 2018/2019 prototypisch umgesetzt. Dabei wurde exemplarisch der Palettentausch herangezogen. Diese Aufgabe stellt eine höhere organisatorische Hürde da als man anfangs vermuten würde. Betrachtet man das einfache Beispiel aus Abbildung \ref{fig:palettentausch}, sieht man einen immer wiederkehrenden Kreislauf. Sinn und Zweck des Palettentausch ist es, immer eins zu eins vollständig leere Europaletten gegen mit Ware beladene zu wechseln. Deshalb muss der Lastkraftwagen zunächst mit leeren Paletten beladen zum Produzenten fahren um dort die Ware zu beladen, welche er sodann zum Kunden transportiert. Dort angekommen erhält er wieder in gleicher Anzahl zur Lieferung leere Paletten. In der Praxis stellt sich das jedoch nicht als realistisches Szenario dar: Bei voller Beladung mit leeren Paletten muss immer noch genügend Platz für unpalettierte Ware vorhanden sein. Dadurch ist ein gleichmäßiger Palettentausch nicht durchsetzbar. Mittels oft nicht standardisierten Papierformularen, sogenannten Palettenscheinen, wird der Austausch zwischen den Geschäftspartnern quittiert. Sollte es zu einer ungleichmäßigen Verteilung kommen, wird ein Kontenbetrag von jedem Teilnehmer selbstständig abgebucht, um einen Ausgleich herzustellen. \cite[vgl.][]{Disponaut2016}

In der Zwischenzeit gab es eine Initiative der GS1, um den Palettenschein zu standardisieren und digitalisieren \cite[vgl.][]{GS12017}. Dadurch soll es auch möglich sein, ihn elektronisch untereinander auszutauschen. Ein Problem aber bleibt; die eigenhändige Abrechnung der beiden Parteien. Hier offenbart sich ein entscheidender Vorteil eines Smart Contracts. Da er die Kontostände der Teilnehmer kennt, kann er an zentraler Stelle die Schulden auf die jeweiligen Konten buchen. Dadurch fällt ein großer Teil der organisatorischen Komplexität weg. So wurde es auch in einem Proof of Concept in der oben genannten Bachelorthesis gezeigt, welcher aber nie im Produktionsmodus angewandt wurde.

Allgemein hat sich bei der Firma Dachser herausgestellt, dass die Blockchain und die darunterliegenden Smart Contracts nicht unter den aktuellen Gegebenheiten als sinnbringende Technologien eingesetzt werden können. Damit es eine praktikable Lösung darstellt, müssten weitere Industriepartner in die Blockchain eingebunden werden, darunter auch direkte Konkurrenten der Firma Dachser, wie beispielsweise DB Schenker. Per Definition kann nun jeder die Transaktionen, welche bei einem Smart Contract anfallen, nachvollziehen. Wollte man die Sichtbarkeit untereinander einschränken, müsste man die Blockchain so abändern, dass sie im Endeffekt nicht mehr ihrer eigentlichen Bestimmung gerecht wird. In der Branche wird ein Umdenken gefordert, damit nützliche Modelle nicht an der Umsetzung scheitern müssen. Letztendlich hängt ein Großteil nicht an der Technik, sondern an der unternehmenspolitischen Ausführung.\\
Ein weiteres Manko ist das Fehlen von Standards bei der Implementierung von Smart Contracts, da sie nicht im Nachhinein vom Programmierer abgeändert werden können. Das sorgt zusehends für Verwirrung und Ungereimtheiten bei der Implementierung, was bei klassischen Systemen durch Industrie- und Programmierstandards nahezu ausgeschlossen werden kann. Oftmals stellt sich eine herkömmliche Lösung als einfacher und tragfähiger heraus, besonders dort, wo sowieso nur ein Server/Node benötigt wird, was die dezentrale Implementierung in einer Blockchain obsolet macht.\\
Betrachtet man die heutigen Anforderungen an den Datenschutz im Unternehmen, so steht die Speicherung in einer Blockchain unter Umständen in klarem Kontrast du den Forderungen der Datenschutzgrundverordnung. Diese fordert eine Löschbarkeit bzw. Anonymisierung der Daten bei Erstellung respektive im Nachhinein. Durch die Unveränderlichkeit der Blockchain ist das schlichtweg nicht durchführbar. Die aktuell vorherrschenden Gesetze bieten keine Möglichkeit, Probleme wie den Palettentausch mit Palettenschein sinnvoll auf einen Smart Contract abzubilden.

\section{Tradelens}
Eine erfolgreich im Markt bestehende Softwarelösung ist Tradelens, ein Produkt, welches in Kooperation von IBM und der dänischen Reederei Maersk entstanden ist. Das Ziel ist, eine Plattform für möglichst viele Beteiligte zu liefern, um damit an der Spitze der digitalen Transformation in der Logistik zu stehen \cite[vgl.][S. 7]{Tradelens2019b}. Tradelens besteht aus insgesamt drei Ebenen: Netzwerk, Plattform und Applikationen (siehe Abbildung \ref{fig:tradelensOverview}).

\begin{figure}[h!]
  \centering
  \includegraphics[width=.4\textwidth]{Bilder/Tradelens-Overview.png}
  \caption[Tradelens Aufbau]{Tradelens Aufbau \cite{Tradelens2019a}}
  \label{fig:tradelensOverview}
\end{figure}

\textbf{Netzwerk}\\
Das Netzwerk bildet die Grundlage von Tradelens. Formal beschrieben umfasst es alle Business-Teilnehmer, welche an dieser spezifischen Blockchain-Lösung teilnehmen. So ist es möglich, die Teilhabenden an der gesamten Supply-Chain abzudecken. Dies reicht von den Verschiffern, über Versandunternehmer bis zu Regierung und das zuständigen Zollbüro. Die anfallende Menge an Informationen und Daten kann genutzt werden, um den gesamten Lebenszyklus einer Lieferung nachvollziehen zu können. \cite[vgl.][S. 5]{Tradelens2019b}

\textbf{Plattform}\\
Plattformseitig zählt Tradelens nicht zu einer Ethereum basierten Lösung. Vielmehr wird auf Hyperledger Fabric aufgesetzt. Das ist eine ursprünglich von IBM entwickelte und an die Linux Foundation gespendete Applikation. Heute existiert mit Hyperledger bereits ein ganzes Ökosystem rund um Software der Blockchain. Diese Plattform bietet vielfältige Möglichkeiten zum Dokumentenaustausch. Wie im vorigen Beispiel bei Dachser angesprochen, lässt sich zeigen, dass noch viele Abläufe der Logistik papierbasiert abgewickelt werden. Tradelens bietet hiermit nicht nur eine hohe Automatisierungsrate für Rechnungen, Packlisten, Buchungsbestätigungen etc., sondern liefert auch standardisierte Formulare. Werden diese Vorgänge abgeschlossen, setzen die Smart Contracts an, um einen reibungslosen Ablauf gewährleisten zu können. Um das Vertrauen der Kunden aufrecht erhalten zu können, wird eine Berechtigungsverwaltung eingesetzt, damit nur an einer Transaktion beteiligte Partner Informationen abrufen können. Zusätzlich sind alle Daten generell verschlüsselt abgelegt. \cite[vgl.][S. 11 ff.]{Tradelens2019b}

\textbf{Applikationen}\\
Auf der letzten Ebene liegen die Applikationen. Tradelens an sich bietet nur einen gewissen Grundstock an Funktionalität. Will man Schnittstellen zu speziellen Systemen herstellen, oder Tradelens um spezifische Fähigkeiten erweitern muss das selbst erfolgen. Daher kann auch jedermann auf einem extra dafür bereitgestellten Marktplatz Applikationen an potentielle Kunden anbieten. Durch diesen Austausch können die Teilnehmer untereinander von den Entwicklungen der anderen profitieren. Tradelens will eine offene Lösung herstellen, welche man selbst nach Belieben erweitern und an seine Vorstellungen anpassen kann. \cite[vgl.][S. 5]{Tradelens2019b}

Versucht man, genauere Informationen über die zugrundeliegende Plattform und die eingesetzten Smart Contracts zu erhalten, kommt man schnell an die Grenzen des Systems. Das ist natürlich einerseits so gewollt, damit die sensiblen Daten der Plattformteilnehmer nicht in fremde Hände geraten, andererseits ist der Grundgedanke einer Blockchain wahrlich ein anderer. Es wird zwar mit einer offenen Plattform geworben -- das mag durchaus legitim sein -- dennoch wird die Open Source Blockchain Hyperledger Fabric so verändert, dass sie nicht mehr einer Blockchain entspricht. Letztendlich kommt es zu einem geschlossenen System, welches in der Hand von einigen wenigen Unternehmen liegt. So ist es kaum verwunderlich, dass ein IBM-Account benötigt wird, um nur grundsätzliche Informationen abzurufen. Das Beobachten von Transaktionen oder gar den Contracts ist schlichtweg als Nichtteilnehmer nicht möglich.

\section{Weitere Beispiele}
In der Logistik bestehen noch weitere prestigeträchtige Beispiele für den Einsatz von Blockchain und Smart Contracts, auf die bisher noch nicht eingegangen wurde. Eines davon ist die Lebensmittel Blockchain des US-Supermarkts Walmart. Dort wurde eine Hyperledger Fabric Lösung mithilfe IBM entwickelt, um eine bessere Nachverfolgbarkeit von Lebensmitteln gewährleisten zu können. Kommt es zu Epidemien aufgrund verseuchter Lebensmittel, lässt sich der Ursprung der Produkte in Sekundenschnelle bestimmen. Ein Vorgang der früher mit bis zu sieben Tagen deutlich komplexer zu bewerkstelligen war. \cite[vgl.][]{Hyperledger2019}\\
Mit Cobility findet sich eine weitere Initiative, um ein dezentrales System in der Logistikbranche zu etablieren. Tiefergehende Informationen sind auch hier leider nicht zu erhalten, da ein geschlossenes System der Firma evan.network verwendet wird. Durch die Unterstützung vieler großer Industriepartner wirkt das System jedoch sehr prestigeträchtig. Innerhalb des zweiten Quartals diesen Jahres wird die erste Anwendung freigeschalten, wozu zuvor noch Rahmenbedingungen definiert wurden. Langfristig soll das System weiter ausgebaut werden und weiter wachsen. Da Cobility ein Gemeinschaftsprojekt dreier deutscher Firmen ist, steht es in direkter Konkurrenz zu Lösungen wie beispielsweise Tradelens. Da der Reifegrad dieser Lösung aber noch nicht so weit fortgeschritten ist, ist es fraglich, inwieweit sie sich gegen Konkurrenzsysteme durchsetzen kann. \cite[vgl.][]{Cobility2019}
\chapter{Reisebranche}
\label{chap:Reisebranche}
Die Reisebranche wird von wenigen Anbietern auf dem Markt bedient, was zu einem sogenannten Oligopol führt. Betrachtet man den größten Reiseportalbetreiber in Deutschland aus dem Jahr 2017, Booking.com, so fällt auf, dass er hierzulande mit weitem Abstand die Liste der umsatzstärksten Unternehmen in diesem Bereich anführt \cite[vgl.][]{FVW2017}. Das ist nicht verwunderlich, vereint die Booking Holding unter sich auch Marken wie Agoda, Momondo oder Kayak \cite[vgl.][]{Booking2019}. Beim zweitgrößten Betreiber Expedia verhält es sich ähnlich; auch hier gibt es Tochterfirmen wie Trivago, Homeaway und Ebookers \cite[vgl.][]{Expedia2019}.\\
Aufgrund des geringen Marktdrucks der Anbieter werden häufig veraltete IT-Systeme eingesetzt, welche zu Sicherheitslücken neigen. Im Jahr 2017 kam es so beispielsweise zu Hackerangriffen auf das Computerreservierungssystem von Sabre, einer der Top drei Firmen auf diesem Gebiet \cite[vgl.][]{Mathews2017}.

Um das Missverhältnis zwischen Angebot und Nachfrage auszugleichen sowie eine Erneuerung der IT anzustreben, engagieren sich Firmen wie Winding Tree, um mittels Blockchain und Smart Contracts eine direkte Verbindung zwischen Airlines bzw. Hotels und den Buchenden herzustellen (siehe Abbildung \ref{fig:windingTreeOverview}). Um die Notwendigkeit dazu zu untermauern hebt Winding Tree in ihrem Whitepaper als Ausgangspunkt vor allem die hohen Gebühren hervor, die die klassischen Intermediäre bei erfolgreicher Vermittlung berechnen \cite[vgl.][S. 2 f.]{WT2019}.

\begin{figure}[h!]
  \centering
  \includegraphics[width=\textwidth]{Bilder/WindingTreeOverview.png}
  \caption[Winding Tree Überblick]{Winding Tree Überblick \cite{WTWebsite2019}}
  \label{fig:windingTreeOverview}
\end{figure}

Winding Tree bietet lediglich die Plattform inklusive Schnittstellen zur Abwicklung der Geschäftsbeziehungen an, alle weiteren Anwendungen (z. B. Benutzeroberflächen) müssen von den Teilnehmern selbst implementiert werden. Dabei stehen als Hilfestellung allerdings auch Referenzimplementierungen in JavaScript unter der Apache-2.0-Lizenz auf Winding Trees GitHub Account bereit \cite{WTGitHub2019}.\\
Dass dies nicht nur eine Nischenlösung ist, wird beim Blick auf die Industriepartner bewusst: Neben Hotelketten wie Nordic Hotels sind vor allem große europäische Airlines auf der Plattform vertreten. Dazu gehören unter anderem auch AirFrance, KLM, SWISS, Eurowings und die Lufthansa \cite{WTWebsite2019}.

Eigens für die Plattform wurde der Líf Token generiert, welcher mehr Daten als ein typischer ERC20 Token verarbeiten kann und dabei die Kompatibilität beibehält \cite[][S. 9]{WT2019}. Dadurch können die für Reisen benötigten Daten kostengünstiger verarbeitet werden.\\ 
Zudem ergibt sich der Nebeneffekt der Plattformfinanzierung, was 2018 in Form eines Token Generation Events umgesetzt wurde. In mehreren Stadien konnten Ether in Líf umgewandelt werden. In der ersten Woche wurden 1000 Líf je Ether ausgeschüttet, in der zweiten Woche noch 900. Dabei wird die Anzahl der generierten Token vom Markt bestimmt. Stand heute (26. Juni 2019) ist ein Líf zum Preis von 0.0003 ETH respektive 0.91831 Eurocent erhältlich \cite{Coinmarketcap2019}. Um die laufenden Kosten für die Entwicklung der gemeinnützigen Firma Winding Tree tragen zu können, wurde zudem ein Marktvalidierungsmechanismus eingebaut. Dieser ist in Form eines autonomen Smart Contracts angelegt, welcher die Kapitalisierung, die die Marke von umgerechnet zehn Millionen US-Dollar zum Token Generation Event überschritten hat, verwaltet. Anhand einer festgelegten Preisfunktion werden Token angekauft und direkt vernichtet. Heute (Stand 26. Juni 2019) werden dort noch 4189 Ether, umgerechnet mehr als eine Million Euro, vorgehalten \cite{Lif2019}. Anhand [vgl.][S. 12 ff.]{WT2019}. \\
Somit gilt Líf als Wertinstrument für alle abzuwickelnden Tätigkeiten, die auf dem System stattfinden sollen. Am besten wird das am Beispiel aus Abbildung \ref{fig:windingTreeExample} deutlich.\\

\begin{figure}[h!]
  \centering
  \includegraphics[width=\textwidth]{Bilder/WindingTreeExample.png}
  \caption[Beispiel eines Geschäftsfalls]{Beispiel eines Geschäftsfalls \cite[S. 10]{WT2019}}
  \label{fig:windingTreeExample}
\end{figure}

Um aus dem Property Management System des Hotels freie Zimmer an Winding Tree zu melden, muss eine zu diesem Zeitpunkt berechnete Menge an Líf gezahlt werden. Betrachet man die Smart Contracts und die API-Implementierungen aus dem oben bereits genannten GitHub Account, wird nicht sofort ersichtlich, wie diese Inventur getätigt wird. Die Contracts für die Airlines und Hotels sind relativ kurz gehalten und beschränken sich auf das Nötigste. Im Fall von Änderungen werden daher in der Regel nur Daten im Byteformat übertragen. Um die Zugriffe auf den Contract zu kapseln, und damit einfacher zu gestalten, dienen die angesprochenen APIs, welche in \emph{booking}, \emph{notification}, \emph{write} und \emph{read} unterteilt sind \cite[vgl.][]{WTAPIOverview2019}.\\
Um solche Änderungen einzupflegen müsste über den Contract zuerst ein Hotel im System angelegt werden. Dazu dient die Methode \texttt{registerHotel(...)}. Diese bindet das Hotel an einen \emph{manager}, der jedoch mittels \texttt{transferHotel(...)} das Hotel an eine andere Adresse übertragen kann. Über die \emph{write}-API werden schlussendlich die Daten des Hotels über die Methode \texttt{callHotel(...)} des Hotels überschrieben. \cite[vgl.][]{WTHotelContract2019, WTInventory2019}\\
Nun ist das Hotel für eine etwaige Reiseagentur über die Plattform auffindbar. Diese kann per \emph{read}-API potentielle Hotels abrufen. Momentan werden die Hotels in der Reihenfolge, wie sie in den Winding Tree Index geschrieben worden sind, ausgegegen. Somit lässt sich auch erst einmal nicht nach Hotels in der gewüschten geographischen Lage filtern, was durchaus problematisch werden könnte, wenn man kein Hotel als möglichen Kandidaten ausschließen will. Jedoch können unnötige Informationen über die Hotels auch mithilfe des Requests entfernt werden, d.h. z.B. lediglich die Blockchain-Adresse und die Koordinaten angezeigt werden, womit aber nicht das Problem des Abrufens letztendlich aller Hotels des Indexes gelöst wird. \cite[vgl.][]{WTQueryInventory2019}\\
Ist ein passendes Hotel gefunden, kann es gebucht werden. Dafür wird die \emph{booking}-API verwendet. Neben den Informationen über Gäste müssen auch Preisinformationen angefügt werden. Diese können aus den Daten der Plattform berechnet werden. Wieso die Daten nicht innerhalb eines Contracts beigefügt werden, ist unklar. \cite[vgl.][]{WTBooking2019}\\
Um auf Hotelseite die Buchung zu akzeptieren, muss eine sogenannte \texttt{bookingUri} hinterlegt sein. Um das einzutragen, sei an dieser Stelle auch wieder auf die \texttt{callHotel(...)}-Methode des dazugehörigen Smart Contracts verwiesen. Die URI zeigt auf ein hotelinternes System, welches eine REST API aufspannt, welche wiederum die Buchungsschnittstelle abbildet. An den Endpunkten der Schnittstelle kann dann eine Buchung hinterlegt, bzw. im Zweifelsfall storniert werden. \cite[vgl.][]{WTBookingAccept2019}

Als Abschluss dieses Kapitels sollen nochmals kurz einzelne Punkte aufgegriffen werden.\\
Positiv anzumerken ist das Verfügbarmachen aller Implementierungen. Von den Contracts bis hin zu den Schnittstellen-Implementierungen kann alles nachvollzogen werden. Bei Bedarf besteht die Möglichkeit, tiefer in das System einzusteigen. Ob jedoch die Öffnung des Quellcodes zu mehr Sicherheit beiträgt, ist umstritten. Die Möglichkeit, auf Schwachstellen aufmerksam machen zu können, ist dennoch hervorzuheben und bekräftigt die Ausrichtung Winding Trees zu offenen Softwaresystemen, wie der Ethereum Blockchain. Durch namhafte Kooperationen und eine im Voraus geplante Finanzierung, ist diese Smart Contract Anwendung durchaus als ernsthaft und positiv zu betrachten.\\ 
Dennoch wird aus den Reihen der Reisebranche auch Kritik laut.

%- Líf-Token -> Notwendigkeit, Grundlage, ganz kurz Token Event; Market Validation Mechanism interessanter [Seite 2 und 3 halb] \\ % https://windingtree.com/White_Paper_EN.pdf
%- Ablauf allgemein (Beispiel aus Whitepaper) + Contracts genauer (siehe GitHub) [Seite 3 halb und 4]\\#
%- Kritik am System [Seite 4] \\
\chapter{Versicherung}
\label{chap:Versicherung}
Im Bereich Versicherung gibt es mehrere Bestrebungen, mithilfe von Smart Contracts die automatische Abwicklung von Versicherungsfällen zu regeln. Die 2018 gegründete B3i Initiative, hinter der namhafte Versicherungsunternehmen wie Zurich, AXA, Generali und die Allianz stehen, hat sich zur Aufgabe gemacht, Standards für die Branche zu entwickeln und das nötige Netzwerk und die Anwendungen aufzubauen \cite[vgl.][]{B3iWhoWeAre2019}. Da die Applikation erst im Juli 2019 startet \cite[vgl.][]{B3iHackathon2019}, eignet es sich nicht, um tiefer darauf einzugehen.

Eine bereits lauffähige Anwendung auf der Ethereum Blockchain existiert mit fizzy, einer Applikation von AXA. Fizzy sichert Personen gegen Flugzeugverspätungen bei mehr als zwei Stunden oder kompletter Annulierung des Fluges ab. Mit Anlegen einer Versicherung, was bis spätestens fünf Tage vor Abflug möglich ist, wird die Fluginformation an einen Smart Contract weitergegeben, welcher die Daten in der Blockchain speichert. Durch eine Koppelung an Flugdaten wird der Betrag im Versicherungsfall innerhalb einer Standard-Bankenlaufzeit ausbezahlt. Die Gebühren für die Absicherung werden dabei von einem Algorithmus anhand des Flug-Ausfallsrisikos berechnet. Durch den Einsatz von Blockchain-Technologie wird auf beiden Seiten Transparenz und Unveränderlichkeit der Bedingungen garantiert. \cite[vgl.][]{Fizzy2019}

Da fizzy mit \cite{Clement2019} einen Artikel bereitgestellt hat, um einen besseren Einblick in den zugrundeliegenden Smart Contract zu bekommen, wird im Folgenden ein Beispiel besprochen.\\
Seit 22. Mai 2019 gibt es eine neue Variante des Smart Contracts \cite{EtherscanNewContract2019}, welcher mit 742 Zeilen Solidity Code deutlich mehr Funktionalität liefert, als der ursprüngliche Smart Contract mit gerade einmal etwas mehr als 200 Codezeilen \cite{EtherscanOldContract2019}. Zur Betrachtung wird nun in Teilen der neue Contract herangezogen.

Die Schritte (1) und (5) aus Abbildung \ref{fig:fizzyAblauf} können vernachlässigt werden, da sie sich nur mit der Eingabe der persönlichen Daten und Flugdetails beschäftigen und der letzendlichen Kompensation im Versicherungsfall dienen. Interessanter sind hier die Schritte (2), (3) und (4), welche sich direkt mit dem Smart Contract beschäftigen. Nach diesen wird jeweils die Funktion \texttt{InsuranceUpdate(...)} im Nachgang aufgerufen.\clearpage

\begin{figure}[h!]
  \centering
  \includegraphics[width=\textwidth]{Bilder/fizzyAblauf.png}
  \caption[Ablauf fizzy Smart Contract]{Ablauf fizzy Smart Contract \cite{Clement2019}}
  \label{fig:fizzyAblauf}
\end{figure}

Anfangs muss eine Versicherung angelegt werden. Dafür steht die Solidity-Funktion \texttt{addNewInsurance(bytes32 flightId, uint256 productId, uint256 premium, uint256 indemnity, uint256 limitArrivalTime, uint256 conditions)} bereit. Kurz zur Erläuterung der einzelnen Parameter:

\begin{itemize}
    \item \texttt{flightId}: In Hexadezimal-Darstellung steht die Flugnummer inklusive Abflugdatum (UNIX-Timestamp) bereit
    \item \texttt{productId}: Identifikator, um Versicherung eindeutig zu kennzeichnen; es können auch mehrere Versicherungen mit der gleichen \texttt{flightId} existieren
    \item \texttt{premium}: Prämienbetrag, hexadezimal codiert
    \item \texttt{indemnity}: Entschädigung, hexadezimal codiert
    \item \texttt{limitArrivalTime}: spätester Ankunftszeitpunkt, d.h. geplanter Ankunftszeitpunkt + 2 Stunden (UNIX-Timestamp, UTC)
    \item \texttt{conditions}: binär kodierte Konditionen; 1001 (2) = 9 (10): für Verspätung und Canceln abgesichert, 0001 (2) = 1 (10): nur für Canceln abgesichert
\end{itemize}

Aus dem Contract wird die Transaktion\\
\texttt{0x6ecc9789935ff5c518f44fa9b2601f40ac6a8a4fa45f16f21146251dc9c9536a} beispielhaft ausgewählt. Die zugehörigen Werte obiger Funktion sind in Abbildung \ref{fig:fizzyNewInsurance} zu sehen.

\begin{figure}[h!]
  \centering
  \includegraphics[width=\textwidth]{Bilder/FizzyExampleNewInsurance.png}
  \caption[Input-Daten addNewInsurance]{Input-Daten addNewInsurance \cite{FizzyNewInsurance2019}}
  \label{fig:fizzyNewInsurance}
\end{figure}

Mithilfe eines Hexadezimal zu String Decoders lässt sich die \texttt{flightId} zu \\ \texttt{AR.1500.1561680000000} umwandeln. Dabei beschreibt \texttt{AR.1500} den Flug, was sich nach einer Suchmaschinen-Recherche als Reise von Buenos Aires nach Córdoba mit der Fluggesellschaft Aerolineas Argentinas herausstellt. \texttt{1561680000000} ergibt mit einem UNIX-Timestamp-Converter den 28. Juni 2019. Auch die Hexadezimalen Werte der Prämien- und Entschädigungsbeträge lassen sich einfach umwandeln. So kommt man auf eine Prämie von 4€ sowie eine mögliche Entschädigung in Höhe von 50€. Die limitierte Ankunfszeit liegt am 28. Juni 2019 um 13:55 (GMT). Der Versicherte hat sich sowohl gegen Verspätung als auch gegen Ausfall abgesichert.

\begin{figure}[h!]
  \centering
  \includegraphics[width=\textwidth]{Bilder/FizzyExampleFlightLanded.png}
  \caption[Input-Daten setFlightLandedAndArrivalTime]{Input-Daten setFlightLandedAndArrivalTime \cite{FizzyFlightLanded2019}}
  \label{fig:fizzyFlightLanded}
\end{figure}

Sieht man in einem Tool wie etherscan.io in die Transaktionsliste, findet man am 28. Juni die Transaktion \texttt{0xcb63f57039f144073c4a4b45621a3c0da4593bd2185c14838d7b7e4c537f5be9}, deren Daten auch in Abbildung \ref{fig:fizzyFlightLanded} zu sehen sind. Da die tatsächliche Ankunftszeit mit umgerechnet 11:47 (GMT) weit unter der Höchstgrenze liegt, wird in diesem Fall keine Kompensation ausgezahlt.\\
Anders im Beispiel aus Abbildung \ref{fig:fizzyTrigger}, bei dem der Status auf 1 gesetzt wird, was gleichzusetzen mit einem abgeschlossenen Flug inklusive Kompensation ist.

\begin{figure}[h!]
  \centering
  \includegraphics[width=\textwidth]{Bilder/FizzyExampleTrigger.png}
  \caption[Triggern einer Bedingung]{Triggern einer Bedingung \cite{FizzyTrigger2019}}
  \label{fig:fizzyTrigger}
\end{figure}

Fizzy ist nur eines von vielen Beispielen der Versicherungsbranche, die auf die Blockchain setzen. Es scheint so, als wäre auch hier ähnlich den Anfangsbestrebungen des B3i eher die Machbarkeit überprüft worden, anstatt alle gewohnten Prozesse auf Blockchain-Anwendungen und Smart Contracts umzustellen. Das belegen auch die Zahlen: Der alte Smart Contract umfasst ca. 20 000 Transaktionen \cite{EtherscanOldContract2019}, der neue noch nicht einmal eine dreistellige Anzahl \cite{EtherscanNewContract2019}. Dabei gilt zu berücksichtigen, dass 20 000 Einträge nicht 20 000 Versicherungen bedeutet, da mit dem Anlegen einer Versicherung und der Landung des Flugs mindestens zwei Methoden des darunterliegenden Contracts aufgerufen werden müssen. Dennoch ist die Umsetzung gelungen, da die Transparenz im Bezug auf die Bedeutung der einzelnen Parameter der Contract-Funktionen sehr positiv auffällt. So kann jeder die Abwicklung seines Fluges beobachten, ohne das persönliche Daten offengelegt werden. Trotzdem stellt sich auch hier wieder die Frage, inwiefern der Einsatz einer Blockchain-Lösung Vorteile bringt. Die Abwicklung der Kompensation kann zwar durch die eingebauten Trigger automatisch von Statten gehen, dennoch wird ein System für das Anlegen der Versicherung inklusive Datenhaltung für persönliche Daten und Zahlungsdaten benötigt. Hinzu kommen noch die Transaktionsgebühren der Ethereum Blockchain.




\chapter{Energieversorgung}
\label{chap:Energieversorgung}
Für die Betrachtung von Smart Contracts in der Energieversorgung, eignet sich das lokal in Kempten ansässige Unternehmen Allgäuer Überlandwerk (AÜW).

\begin{wrapfigure}{r}{0.5\textwidth}
\centering
  \includegraphics[width=.4\textwidth]{Bilder/Microgrid.png}
  \caption[Microgrid]{Microgrid \cite{AÜW}}
  \label{fig:Microgrid}
\end{wrapfigure}

Eines der dort durchgeführten Projekte ist das sogenannte \emph{Allgäu Microgrid} aus Abbildung \ref{fig:Microgrid}. Ziel dieses Grids (engl., hier gemeint: Stromnetz) ist es, eine Plattform für den Peer to Peer Stromhandel zu entwickeln. Dafür wurde die Partnerschaft mit LO3-Energy aus New York gesucht, welche mit dem Brooklyn Microgrid bereits eine ähnliche Plattform aufgebaut haben \cite[vgl.][]{BrooklynGrid2019}. Welche Blockchain-Lösung dafür eingesetzt wird, wird nicht erwähnt. Im Allgäu soll das Grid vor allem auf die Zeit nach dem Erneuerbare Energien Gesetz abzielen, d.h. wenn die staatliche Förderungen für Votovoltaikanlagen so weit zurück geht, dass eine Rentabilität nicht mehr gegeben ist. Deshalb wird mithilfe von fünf Pilotkunden eine Plattform in der Ortschaft Wiggensbach im Allgäu aufgebaut, welche zum Stromhandel dient. Daran sind sowohl Anlagenbesitzer (Prosumer, Producer + Consumer) als auch reine Konsumenten (Consumer) beteiligt. Fällt bei den Prosumern überschüssige Energie an, kann diese über das Netzwerk an Andere verkauft werden. Bisher wird der Strom wieder beim Energieversorger eingespeist, wofür das Erneuerbare-Energien-Gesetz die Grundlage bildet. Aufgrund von fallender Bezuschussung wird der Rückerverkauf von Strom aber immer unrentabler. Bei dem Microgrid hingegen können beide Seiten einen Schwellwert festlegen, bei dem verkauft bzw. angekauft werden soll. Kommt es zu einem Matching von Angebot und Nachfrage mithilfe eines Smart Contracts, wird die Umverteilung in die Wege geleitet. Um das bewerkstelligen zu können, werden einige technische Grundvoraussetzungen gestellt. Dazu zählt unter anderem auch ein DSL-Anschluss mit einer Übertragungsleistung von 16 000 MBit/s. Des Weiteren benötigen sowohl Prosumer als auch Consumer ein Messsystem, welches die Daten über den Verbrauch des Stroms bzw. dessen Erzeugung an die Plattform übermittelt. Dieses wird vom AÜW kostenlos zur Verfügung gestellt. Um den bereits genannten Ein- bzw. Verkaufswert angeben zu können, wird eine Smartphone-/Tablet-App benötigt. Daher sollen vor allem technik-affine Menschen angesprochen werden. \cite[vgl.][]{AÜW, Klaus2018}

Dieses bereits abgeschlossene Projekt dient als Vorlauf für das seit 2018 drei Jahre andauernde Forschungsprojekt \emph{Pebbles}. Daran sind neben dem AÜW auch die Hochschule Kempten, das Fraunhofer Institut, Siemens und Allgäu Netz beteiligt. Hier sollen die Möglichkeiten eines Peer to Peer Stromhandels weiter erforscht werden, z.B. indem auch Firmen an die Plattform und somit auch an die Blockchain-Lösung angeschlossen werden. Auch die bisher eingesetzen Smart Contracts sollen ausgeweitet werden, um ein breiteres Funktionssprektrum anbieten zu können. So soll neben einer lokalen Umverteilung des Stroms auch an einer Strombörse Handel betrieben werden. Dazu kommen weitere Applikationen, z.B. für Data Mining, welche die Daten der Plattform weiterverarbeiten können. \\
Hürden, wie die in anderen Bereichen bereits angesprochene DSGVO soll auch zu einer ganzheitlichen Klärung der Thematik beitragen. Nach den theoretischen Vorarbeiten sollen die Konzepte auch schlussendlich simuliert werden, um die Praxistauglichkeit überprüfen zu können. \cite[vgl.][]{Ziegler2018}

Die angestrebten Plattformen des AÜW sind nur ein Beispiel für den Fortschritt in der Digitalisierung und dem damit involvierten Einsatz von Technologien wie Smart Contracts. Auch in anderen Städten in Deutschland wird bereits an Ähnlichem geforscht; in anderen Ländern wie beispielsweise Norwegen ist der smarte Umgang mit Energie und Elektromobilität bereits Alltag. \\
Leider war es auch in diesem Beispiel nicht möglich, Einblicke in die Smart Contracts zu erhalten, was sich aber unter Anbetracht der gewünschten Sicherheitsaspekte bei Energieplattformen durchaus verständlich ist. Zudem stellen solche innovativen Ansätze Vorteile gegenüber der Konkurrenz da, weshalb oftmals keine Öffnung des Programmcodes stattfindet. \\
Inwiefern sich Vorteile durch den Einsatz von Blockchains und Smart Contracts ergeben, wird zudem durch Pebbles von verschiedenen Seiten beleuchtet. Daher wird sich zeigen, ob solche neuen Plattformen die gewünschten Effekte und Ansprüche erfüllen können.
\chapter{Hilfsgelder}
\label{chap:Hilfsgelder}
Zum Abschluss der Beispiele soll eine Anwendung von Smart Contracts aufgezeigt werden, die sich grundlegend von den anderen bereits vorgestellten unterscheidet. Wurde bisher immer von einem Business-Use-Case ausgegangen, können auch komplett unterschiedliche Felder von Smart Contracts profitieren. So auch die Verteilung von Hilfsgeldern, wie sie unter anderem von der UN gepflegt wird. 

Eines der Felder der Vereinten Nationen ist die weltweite Hungerbekämpfung mithilfe des \emph{World Food Programme (WFP)}. Die Zahlen zeugen von dem enormen Ausmaß der Hilfeleistungen: So wurden im Jahr 2018 über das WFP mehr als 1,6 Milliarden US-Dollar an Hilfsgeldern in Form von Bargeld an Bedürftige vergeben. Zuvor wurden in größeren Mengen Sachgüter vergeben, was jedoch auf der einen Seite keinen Beitrag zur dortigen Wirtschaft leistet, als auch die Flexibilität der Hilfeempfänger einschränkt. Im Umkehrschluss muss dabei seitens der Organisation auch dem Missbrauch der Gelder vorgebeugt werden. So besteht die Gefahr, dass über Mehrfachidentitäten unberechtigt Geld erschlichen werden kann. Auch auf Seiten der Banken ist man in instabilen Regionen dieser Erde nicht vor Unterschlagungen geschützt. Um diese Faktoren ausschließen zu können, wird seit 2017 in dem Projekt \emph{Building Blocks} jede Auszahlung über die Ethereum Blockchain authentifiziert und registriert. Sollten die erhaltenen und die zustehenden Leistungen divergieren, können daher Unregelmäßigkeiten aufgedeckt und dem Missbrauch von Spenden entgegengewirkt werden. Darüber hinaus ergeben sich nach eigenen Angaben der UN Kosteneinsparungen in Höhe von bis zu 98 \%. Das resultiert unter anderem aus dem Wegfall der an den Transaktionen beteiligten Intermediären, wie beispielsweise Banken. Dadurch kommt auch ein höherer Prozentsatz der Gelder direkt bei den Betroffenen an. Eine Verfeinerung der Technologie ergibt sich mit der Hinzunahme von Iris-Daten, wie sie als eindeutige Identifizierungsmöglichkeit an immer mehr Stellen eingesetzt wird (siehe Abbildung \ref{fig:IrisSystem}). Davor konnten Hilfsbedürftige mittels an Geldautomaten installierten Iris-Scannern, ohne PIN-Code oder sonstige Identifikation, Bargeld abheben. Jetzt können Lebensmitteleinkäufe komplett bargeldlos und sicher auf der Blockchain abgewickelt werden. Nach ersten Tests in Pakistan wurde das System auf Flüchtlingsunterkünfte in Jordanien ausgeweitet. Hier weichen die Zahlen ab; in einer Quelle ist von 10 000 Personen die Rede, in der anderen um einen Faktor zehn multiplizierten Wert. Nichtsdestotrotz zeigt dieses Beispiel, dass es sich hierbei um größere Maßstäbe handelt. \cite[vgl.][]{WFP2019, WFPBlockchain2018}

\begin{figure}[h!]
  \centering
  \includegraphics[width=\textwidth]{Bilder/Iris-Scan.jpg}
  \caption[Iris-System der UN]{Iris-System der UN \cite{WTWebsite2019}}
  \label{fig:IrisSystem}
\end{figure}

Wie das System im Hintergrund genau funktioniert und mit der im Detail Blockchain interagiert, ist unklar. Dies dürfte aber auch der Identitätssicherung der Flüchtlinge und Hilfeempfänger geschuldet sein. Dabei gilt es anzumerken, dass das Iris-System offline arbeitet und die Einträge in der Blockchain vermutlich nur über Identifikationsnummern realen Personen zugeordnet werden können. Wie viele Smart Contracts eingesetzt werden, lässt sich auch nur spekulieren. Bei der Abprüfung von zustehenden Leistungen ist aber die Nutzung eines Contracts wahrscheinlich, ähnelt sie doch beispielsweise der Kontostandsabfrage aus dem Dachser-Beispiel.\\
Wieviel an den Transaktionskosten in absoluten Zahlenwerten gespart wird, wird von der UN nicht genannt. Dies wäre eine interessante Kennzahl gewesen, betrachtet man die stark schwankenden Transaktionskosten für das Mining bei Kryptowährungen und dem Bestätigen von Transaktionen. \\
In Anbetracht der desolaten Lage in den Flüchtlingsgebieten wird der Sicherheitsaspekt durch die Nachvollziehbarkeit aller einzelnen Transaktionen und deren Zuordnung zu Personen verständlich. Dass so etwas aber in Ländern wie Deutschland Datenschutzbedenken hervorrufen würde, gilt als wahrscheinlich.

%Mehr Sicherheit für UN wegen Missbrauch - Datenschutzbedenken wären vor allem in Deutschland, Iris-System nicht an das Internet angeschlossen


\chapter{Fazit und Ausblick}
\label{chap:FazitUndAusblick}


\pagenumbering{Roman}
%Literaturverzeichnis
\printbibliography
\appendix
%Abbildungen, die nicht im Text auftreten
%\include{Abbildungen} 
\include{Anmerkungen}
% Selbständigkeitserklärung
\include{Erklaerung} 

%\renewcommand{\appendixtocname}{Anhang}
%\renewcommand{\appendixname}{Anhang}
%\renewcommand{\appendixpagename}{Anhang}
%\input{latexhints-german}
%\input{latexhints-minted-german}

\pagestyle{empty}
\renewcommand*{\chapterpagestyle}{empty}
\end{document}
