\chapter{Fazit und Ausblick}
\label{chap:FazitUndAusblick}
Blockchain und Smart Contracts sind zweifelsohne eine sehr spannende Thematik. Die Bewertung, ob sie allerdings die \enquote{Jahrhundertechnologie} ist, wird offen gelassen und anderen Personen überlassen. Ähnlich wurde in den 80er- und 90er-Jahren des letzten Jahrtausends die Objektorientierung als das Wunderwerk und die Lösung für alle Probleme gesehen. Das erinnert sehr an das Prinzip \emph{Golden Hammer}, wie es der US-amerikanische Psychologe Abraham Maslow in den 60-Jahren beschrieben hat: Eine Lösungsweg der scheinbar überall anwendbar ist. Ob Blockchain und Smart Contracts also den Hype überstehen, lässt sich sowieso erst im Rückblick beurteilen.

Wie an den Beispielen klar wurde, gibt es viele Bereiche im Business to Business und Business to Consumer Bereich, in denen Smart Contract eingesetzt werden können. Manches, wie beispielsweise Tradelens, wurde bereits mit großen Skalierungsfaktoren etabliert. Bei diesem Beispiel ist allerdings in der Art der Umsetzung als geschlossenes System mit der eigentlichen Blockchain-Idee, wie sie Satoshi Nakamoto beschreibt, keine große Überschneidungsmenge mehr zu finden.

Schlussendlich ist die Technologie nur eine zu betrachtende Komponente der Gleichung. An anderer Stelle steht die rechtliche Grundlage von Smart Contracts wie sie Stand heute implementiert sind. Um der Rechtssprechung verschiedener Länder zu entsprechen, müssen die Contracts jeweils angepasst werden. Das heißt im Umkehrschluss, dass hier neben Programmierern auch Juristen bei der Implementierung herangezogen werden sollten. Darüber hinaus darf auch der Datenschutz nicht ins Hintertreffen geraten, da in der Blockchain abgespeicherte Daten unwiderruflich dort gespeichert sind. Bei einer neuen Rechtslage könnten so gewisse Forderungen aufgrund der Architektur nicht umgesetzt werden. Generell sind Regierungen bei der Bearbeitungen informationstechnischer Themen träge, was sich mit der Schnelllebigkeit einer globalisierten Internetstruktur kaum verhindern lässt. Dennoch fordern neue Themen schnelle Bearbeitung, um in immer mehr Fällen Rechtssicherheit herstellen zu können.

Spielen mehrere dieser Faktoren zusammen, könnten Smart Contracts ein neues Zeitalter im Bezug auf Verträge einläuten und ein Paradebeispiel für die Digitalisierung darstellen. So war es z.B. mit der Programmiersprache Java in den 90er-Jahren für die Objektorientierung, welche auch heute noch einen hohen Stellenwert in Unternehmen weltweit inne hat.