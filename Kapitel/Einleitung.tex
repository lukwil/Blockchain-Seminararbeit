\chapter{Einleitung}
\label{chap:Einleitung}
Die Blockchain-Technologie hat sich in den letzten Jahren zu einem regelrechten Hype-Thema entwickelt. So ist es nicht verwunderlich, dass alleine in Berlin im Jahr 2018 bereits 64 Blockchain-Startups niedergelassen waren \cite{AnzahlStartups2018}. Kryptowährungen, wie beispielsweise Bitcoin oder Ethereum, sind meist die erste Assoziation, die der Begriff Blockchain hervorruft. Über deren Sinn und Daseinsberechtigung im Geldmarkt wird seit Beginn angeregt diskutiert. So sagte der amerikanische Großinvestor Warren Buffet 2014 in einem Interview mit dem US-amerikanischen Sender CNBC über Bitcoin: \qq{Stay away from it. It’s a mirage, basically} \cite{Buffett2014}.

Es muss auch angemerkt werden, dass die Blockchain längst nicht mehr ausschließlich für Kryptowährungen bereitsteht. Mit der kontinuierlichen Weiterentwicklung avancierten vor allem die sogenannten \textit{Smart Contracts} zu einem wichtigen Bestandteil der Plattformen \cite[vgl.][S.~8]{Schuette2017}. Diese sind vor allem im Geschäftsleben von großem Interesse.

Werden in einem realen Umfeld Absprachen getroffen, stehen für beide Vertragsparteien Forderungen und zu erbringende Leistungen im Vordergrund. In bestimmten Fällen kann nicht auf deren Einhaltung vertraut werden, wodurch sich der Einsatz von Smart Contracts anbietet. Dadurch wird einerseits durch die technologische Architektur der Blockchain Manipulationssicherheit garantiert, als auch die Nachvollziehbarkeit aller geschäftlichen Transaktionen gewährleistet.

Einige Szenarien sollen in dieser Arbeit beispielhaft anhand von Use Cases der vorgestellten Unternehmen dargelegt werden. 